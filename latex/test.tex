\documentclass{article} % Specifies the document class
\usepackage{amsmath, amssymb} % Packages for math symbols and equations
\usepackage{bm} % For bold math symbols

\begin{document} % Start of the document

\section*{Vector Spaces and Linear Transformations}

A vector space is a set of vectors that can be added together and scaled by real numbers. The operations of 
addition and scalar multiplication are defined as follows: given two vectors $\mathbf{u}$ and $\mathbf{v}$, the sum of 
the two vectors is defined as $\mathbf{u} + \mathbf{v}$, and for a scalar $c$, the product of the vector 
$\mathbf{u}$ and the scalar $c$ is defined as $c\mathbf{u}$.

The properties of a vector space are:

\begin{enumerate}
    \item \textbf{Commutativity of Addition}: The order in which we add vectors does not matter.
    \item \textbf{Associativity of Addition}: When we add three vectors, it doesn't change whether we do it first or second.
    \item \textbf{Existence of Additive Identity}: There exists a zero vector that, when added to any other vector, leaves the 
    latter unchanged.
    \item \textbf{Existence of Additive Inverse}: For each vector in the space, there is a vector such that their sum is the zero 
    vector.
\end{enumerate}

Linear transformations are functions between vector spaces that preserve the operations of addition and scalar 
multiplication.

If $\mathbf{T}$ is a linear transformation from a vector space $\mathcal{V}$ to another vector space $\mathcal{W}$, then 
for any two vectors $\mathbf{x}$ and $\mathbf{y}$ in $\mathcal{V}$ and any real numbers $a$ and $b$, the following equation holds: 
\[
\mathbf{T}(a\mathbf{x} + b\mathbf{y}) = a\mathbf{T}(\mathbf{x}) + b\mathbf{T}(\mathbf{y}).
\]
This equation is true for all vectors in the domain space.

Eigenvalues and eigenvectors are used to describe how a linear transformation stretches or shrinks a vector. The 
eigenvalue of an eigenvector represents the amount by which it is scaled under a particular linear transformation.

\end{document} % End of the document

