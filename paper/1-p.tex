\section{Vector Spaces}

\label{sec:vector-spaces}
\begin{equation}
\mathbf{x} \in \mathbb{R}^n
\end{equation}
defines a vector space $\mathbb{R}^n$ as the set of all possible linear combinations of $n$ linearly independent vectors.

The dimensionality of a vector space is defined as the number of linearly independent basis vectors required to span the entire space. A basis for a vector space is a set of linearly independent vectors that spans the entire space.

\label{def:span}
Given a subset $S \subseteq \mathbb{R}^n$, the span of $S$ is defined as:

$$
\text{Span}(S) = \left\{
\sum_{i=1}^{k} c_i \mathbf{x}_i \mid c_i \in \mathbb{R}, k \leq n
\right\}
$$

where $\mathbf{x}_i$ are linearly independent vectors in $S$.

A subspace of a vector space is defined as a subset that satisfies the following properties:

1. The zero vector belongs to the subspace.
2. The subspace is closed under addition.
3. The subspace is closed under scalar multiplication.

\label{def:subspace}
Given a linear subspace $V$ and a set of vectors $\mathbf{x}_i$, we can define the basis for $V$ as:

$$
B(V) = \left\{
\mathbf{x} \in V \mid \text{$\mathbf{x}$ is linearly independent from other vectors in $V$}
\right\}
$$

In conclusion, this chapter provides a comprehensive introduction to vector spaces, including definitions, properties, and basic operations. It covers the concept of dimensionality, basis, linear independence, span, and basis for subspaces.

\label{sec:vector-spaces-2}