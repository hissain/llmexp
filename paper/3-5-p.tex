\section*{Vector Spaces}

This chapter introduces vector spaces, including definitions, properties, and basic operations. It covers concepts of dimensionality, basis, linear independence, span, and basis for subspaces.

The definition of a vector space is given by:

\[
 \mathbf{V} = \left\{ \mathbf{x} \in \mathbb{R}^n : \mathbf{x} + \mathbf{y} \in \mathbf{V}, c\mathbf{x} \in \mathbf{V}, \text{ for all } \mathbf{x}, \mathbf{y} \in \mathbf{V} \text{ and } c \in \mathbb{R} \right\}
\]
where $\mathbb{R}^n$ is the set of all $n$-dimensional vectors.

The dimensionality of a vector space can be defined as:

\[
\dim(\mathbf{V}) = \text{number of basis vectors in }\mathbf{V}
\]

A basis for a vector space $\mathbf{V}$ is a set of linearly independent vectors that span the entire space. The dimension of a vector space is equal to the number of elements in any basis.

Linear independence of vectors means that none of the vectors can be expressed as a linear combination of the others.

The span of a vector space $\mathbf{V}$ is the set of all possible linear combinations of the vectors in $\mathbf{V}$.

A subspace of a vector space $\mathbf{V}$ is a subset of $\mathbf{V}$ that also satisfies the properties of a vector space. The basis for a subspace can be found by identifying a subset of basis vectors from the original vector space.

For example, consider the vector space $\mathbb{R}^2$ with standard basis vectors $\begin{pmatrix} 1 \\ 0 \end{pmatrix}$ and $\begin{pmatrix} 0 \\ 1 \end{pmatrix}$. The span of these two vectors is all of $\mathbb{R}^2$, since any vector in $\mathbb{R}^2$ can be expressed as a linear combination of the two basis vectors.

\[
\dim(\mathbf{V}) = n
\]
\[
\text{span}(\mathbf{V}) = \left\{ c_1\begin{pmatrix} 1 \\ 0 \end{pmatrix} + c_2\begin{pmatrix} 0 \\ 1 \end{pmatrix} : c_1, c_2 \in \mathbb{R} \right\}
\]
\[
\text{basis for }\mathbf{V} = \left\{ \begin{pmatrix} 1 \\ 0 \end{pmatrix}, \begin{pmatrix} 0 \\ 1 \end{pmatrix} \right\}
\]