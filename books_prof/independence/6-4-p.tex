\section{India's Role in Supporting Bangladesh}

India played a crucial role in supporting Bangladesh during its struggle for independence. India provided significant military and economic support to the Bangladeshi rebels, which helped galvanize international opposition to Pakistan's actions.

In 1971, India was facing a crisis of its own, with the Indira Gandhi-led government facing internal dissent and external pressures from China and the Soviet Union. However, when the situation in East Pakistan turned violent, India shifted its focus to supporting Bangladesh. The Indian government extended diplomatic support to Bangladesh, urging international organizations to condemn Pakistani actions.

The Indian military also contributed significantly to the crisis. A contingent of 50,000 Indian troops was sent to East Pakistan to protect Bengali refugees and help restore order. These troops played a key role in preventing further atrocities against the Bangladeshi population.

Furthermore, India provided significant economic support to Bangladesh. The Indian government allocated $250 million in aid to Bangladesh, which helped stabilize the country's economy during its early years of independence.

The impact of India's support on the outcome of the crisis cannot be overstated. Without Indian military intervention and diplomatic pressure, it is unlikely that Pakistan would have been forced to release Bangladeshi prisoners or allow a referendum on the future of East Pakistan.

Overall, India's role in supporting Bangladesh during its struggle for independence was instrumental in shaping the country's history. Today, the two countries enjoy close diplomatic and economic ties, reflecting their shared values and histories.

\section*{Timeline of Key Events}

- 1971: India sends a contingent of 50,000 troops to East Pakistan.
- 1971: India allocates $250 million in aid to Bangladesh.
- 1971: The Indian government urges international organizations to condemn Pakistani actions.
- 1972: Bangladesh gains independence from Pakistan.