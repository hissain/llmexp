The Leadership of A.K. Fazlul Huq
================================

A.K. Fazlul Huq was a key figure in the Bengal National Congress and a prominent advocate for Bengali rights within the Indian Empire.

Early Life and Career
---------------------

Fazlul Huq was born on February 2, 1873, in Dhaka, Bengal Presidency (now Bangladesh). His father, Muhammad Hussain Hafiz, was a local judge. Fazlul Huq studied law at the University of Dhaka and later moved to England for further education.

Fazlul Huq became involved in politics during his time as a student in Dhaka. He joined the Indian National Congress in 1895 and quickly rose through the ranks, becoming one of the most influential leaders of the organization in Bengal.

Leadership and Advocacy
----------------------

As a member of the Bengal Provincial Congress Committee, Fazlul Huq played a crucial role in advocating for Bengali rights within the Indian Empire. He was a strong advocate for greater autonomy for Bengal and often clashed with British colonial officials who resisted his demands.

Fazlul Huq also played a key role in organizing several major protests and strikes throughout his career, including the 1920 Non-Cooperation Movement and the 1931 Quit India Movement.

Later Life and Legacy
---------------------

After India gained independence from British rule in 1947, Fazlul Huq became a prominent leader of the Awami League, a party he co-founded with Muhammad Ali Jinnah, the founder of Pakistan. He served as president of the league until his death in 1966.

Fazlul Huq's legacy is complex and multifaceted. While he was a key figure in the Indian independence movement, he also played a significant role in shaping the country's early politics and policies. His advocacy for Bengali rights helped to lay the groundwork for Bangladesh's eventual independence from Pakistan in 1971.

Fazlul Huq passed away on February 22, 1966, at the age of 93. Despite his passing, his legacy continues to be celebrated and studied by scholars and historians today.