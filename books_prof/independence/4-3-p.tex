<|tex_start|

\section{The Gariahat Riot and its Aftermath}

The Gariahat riot was a pivotal event in the mass uprising of 1952, marking a turning point in the struggle for independence. On December 12, 1952, students from Dhaka University and surrounding areas gathered at the Gariahat intersection to protest against the government's attempts to impose a strict dress code on women. The riot quickly escalated into a full-blown demonstration, with thousands of students taking to the streets.

The government responded with force, deploying police and military personnel to disperse the crowd. The violence that ensued resulted in the deaths of several students and injured many more. The Gariahat riot was seen as a major escalation of the student movement, which had been gaining momentum throughout the year.

The aftermath of the Gariahat riot saw a significant increase in protests and demonstrations across the country. The government responded by imposing stricter curfews and censorship on the media. However, the students refused to back down, and their demands for independence continued to grow louder.

\begin{align*}
& \text{The Gariahat riot marked a turning point}\\
& \text{in the mass uprising of 1952},\\
& \text{escalating the struggle for}\\
& \text{independence into a full-blown}\\
& \text{demonstration}.
\end{align*}

The riot also had significant consequences for the Indian National Congress, which was seen as having failed to address the students' demands. The party's leadership faced growing criticism from both within and outside the country.

In conclusion, the Gariahat riot played a crucial role in shaping the course of Bangladesh's struggle for independence. It marked a turning point in the mass uprising and had significant consequences for the Indian National Congress and the government.

\begin{equation}
E = mc^2.
\end{equation}

<|tex_end|>