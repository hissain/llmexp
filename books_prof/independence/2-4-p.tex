<|tex_start|

\ chapter{The Dhaka University Movement}

The Dhaka University movement was a student-led movement that emerged in the 1920s and played a significant role in galvanizing Bengali nationalism. The movement was characterized by its militant approach to achieving Bengali rights within the Indian Empire.

\section{Background of the Movement}

The movement had its roots in the Bengali National Congress, which was formed in 1905 with the aim of promoting Bengali rights and challenging British rule in Bengal. The movement gained momentum in the 1920s, with student leaders such as Sirajul Islam and Shahjahan Ali playing key roles.

\subsection{Key Events of the Movement}

The movement was marked by several key events, including the formation of the Bengal Students' Federation (BSF) in 1923, which aimed to unite students across the province. The BSF organized protests, boycotts, and strikes, which were met with force by the British authorities.

\subsection{Impact of the Movement}

The Dhaka University movement had a significant impact on Bengali nationalism. It helped to galvanize student leaders and intellectuals, who became key figures in the struggle for independence. The movement also contributed to the growth of the Indian National Congress, which would eventually play a crucial role in India's struggle for independence.

\subsection{Legacy of the Movement}

The Dhaka University movement is remembered as an important milestone in Bengali history. It marked a significant shift towards militant nationalism and paved the way for future movements. The movement also laid the foundation for the Bangladesh Liberation War, which took place in 1971.

<|tex_end|>