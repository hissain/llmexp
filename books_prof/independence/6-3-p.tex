\pagestyle{empty}

\begin{document}

\section{Pakistani Military Crackdown and Mass Exodus}

In March 1971, Pakistani forces launched a brutal crackdown on protesters in East Pakistan, leading to the mass exodus of refugees into India. This event marked a turning point in the Bangladesh Liberation War, drawing international condemnation for Pakistan's human rights abuses.

The Pakistani military regime, led by General Yahya Khan, responded violently to the growing independence movement in East Pakistan, using tactics such as forced disappearances, torture, and extrajudicial killings. The use of these brutal methods sparked widespread outrage and further galvanized opposition to the government.

The mass exodus of refugees into India was a direct result of the Pakistani military crackdown. Thousands of Bangladeshis were forced to flee their homes, leaving behind everything they owned, in search of safety and refuge in neighboring India. This influx of refugees put a significant strain on India's resources and infrastructure, further straining already tense relations between the two countries.

The international community also condemned Pakistan's actions, with many countries, including the United States, imposing economic sanctions and diplomatic isolation on the country. The United Nations Security Council passed Resolution 375 in March 1971, calling for an immediate ceasefire and condemnation of Pakistan's "repugnant" actions.

\end{document}