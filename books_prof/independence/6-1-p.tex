\ chapter{Early 1970s: The Turn of Events}

The period from January to June 1971 saw a significant escalation in protests and demonstrations across East Pakistan, with widespread strikes and marches drawing attention from international media. The student movement, which had been gaining momentum since the mid-1960s, reached its peak in this period.

\section{Escalation of Protests}

The protests were sparked by a combination of factors, including economic inequality, cultural differences between East and West Pakistan, and the growing sense of Bengali nationalism. The Pakistani government, led by General Yahya Khan, responded with force, leading to further unrest and violence.

\subsection{Growing International Pressure}

As the situation in East Pakistan deteriorated, international pressure mounted on the Pakistani government to resolve the crisis. The United States, in particular, was concerned about the potential consequences of a war between India and Pakistan over Kashmir.

\ subsection{Mujib's Leadership}

In this context, Mujibur Rahman, the leader of the Awami League, emerged as a key figure. His leadership and vision for a united Bengali nation resonated with many across East Pakistan, and he became a symbol of hope for those seeking independence from Pakistan.