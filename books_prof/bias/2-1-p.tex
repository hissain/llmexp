\section{What are Cognitive Biases?}

Cognitive biases refer to systematic errors in thinking and decision-making that arise from mental shortcuts, rules of thumb, or heuristics. These biases can lead to inaccurate judgments and flawed reasoning.

According to the Dual-Process Theory (Kahneman & Tversky, 1979), there are two types of cognitive processes: System 1 and System 2. System 1 is fast, intuitive, and automatic, while System 2 is slow, deliberate, and effortful. Cognitive biases often arise from the overreliance on System 1, leading to suboptimal decision-making.

Some common examples of cognitive biases include:

- Availability heuristic: judging the likelihood of an event based on how easily examples come to mind
- Anchoring bias: relying too heavily on the first piece of information encountered when making a decision
- Hindsight bias: believing, after an event has occurred, that it was predictable

These biases are often unconscious and can have significant consequences in various fields, such as finance, healthcare, and social sciences.

References:
Kahneman, D., & Tversky, A. (1979). Prospect theory: An analysis of decision under risk. Econometrica, 47(2), 263-292.