\section{Heuristics: Mental Shortcuts}
Heuristics are mental shortcuts that simplify complex decisions by relying on rules of thumb or general principles. While heuristics can be helpful, they can also lead to cognitive biases if not used critically.

Some common heuristics include:

- Availability heuristic: This heuristic relies on how easily examples come to mind, rather than the actual probability of an event.
- Representativeness heuristic: This heuristic involves judging the likelihood of an event based on how closely it resembles a typical case, rather than on the actual probabilities.
- Anchoring effect: This heuristic occurs when initial values or anchors influence subsequent judgments.

These heuristics can lead to cognitive biases if not used critically. For example, the availability heuristic can cause people to overestimate the likelihood of rare events, while the representativeness heuristic can lead to flawed judgments about groups based on limited information.